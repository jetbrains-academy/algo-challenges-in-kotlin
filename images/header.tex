\usepackage{textcomp}
\usepackage{amsopn, amsmath, amssymb}
\usepackage{booktabs}
\usepackage{etoolbox}

\newcommand{\myproc}[1]{\text{\textsc{\text{#1}}}}
\newcommand{\myvar}[1]{\text{\sl #1}}

\colorlet{mcfill}{mc!20!white}
\colorlet{hcfill}{hc!20!white}
\colorlet{onefill}{one!20!white}
\colorlet{twofill}{two!20!white}
\colorlet{threefill}{three!20!white}
\colorlet{fourfill}{four!20!white}

\newcommand{\myalert}[1]{\textcolor{hc}{#1}}

\usepackage{tikz}
\usepackage{tikz-qtree}
\usepackage{pgfplots}

\usetikzlibrary{
	positioning, 
	matrix, 
	calc,
	arrows,
	shapes,
	fit,
	decorations,
	decorations.pathmorphing
}

\tikzset{
	every picture/.style={
		line width=.4mm,
		text=mc,
		draw=mc,
	},
}

\tikzstyle{arraynodes} = [
	minimum size=5mm, 
	inner sep=0mm,
	anchor=center,
	draw=mc,
	text=mc
]

\tikzstyle{array} = [
	draw=none,
	matrix of nodes,
	nodes={arraynodes},
	ampersand replacement=\&,
	line width=.3mm,
	row sep=-\pgflinewidth,
	column sep=-\pgflinewidth,
]

\tikzstyle{h}=[arraynodes, fill=hcfill]

\newcommand{\logo}{
	\path[use as bounding box] (-2.6, -2.6) rectangle (2.6, 2.6);
	\draw[rounded corners] (-2.5, -2.5) rectangle (2.5, 2.5);
}

\newenvironment{pseudocode}
	{\hrule\medskip}
	{\medskip\hrule}
	
\newcommand{\network}[1]{
	\begin{scope}[label distance=-1mm]
		\foreach \x/\y/\t/\l in {0/1/1/s, 1/2/2/a, 1/0/3/b, 2/1/4/c, 3/2/5/d, 3/0/6/e, 4/1/7/t}
			\node[circle, draw, inner sep=0mm, minimum size=5mm, label=below:$\l$] (\t) at (\x,\y) {};
		\foreach[count=\n] \l in {#1}
			\node at (\n) {\textcolor{mc}{$\l$}};
		\foreach \f/\t in {1/2, 1/3, 2/4, 2/5, 3/4, 3/6, 4/5, 4/7, 5/7, 6/7}
			\draw[->] (\f) -- (\t);
	\end{scope}
}

\newcommand{\spnetwork}[1]{
	\begin{scope}[xscale=1.4, yscale=1.4, label distance=-1mm]
		\foreach \x/\y/\t/\l in {0/1/1/s, 1/2/2/a, 1/0/3/b, 2/1/4/c, 3/2/5/d, 3/0/6/e, 4/1/7/t}
			\node[circle, draw, inner sep=0mm, minimum size=6mm, label=below:$\l$] (\t) at (\x,\y) {};
		\foreach[count=\n] \l in {#1}
			\node at (\n) {\textcolor{mc}{$\l$}};
		\foreach \f/\t/\w in {1/2/4, 1/3/5, 2/4/4, 2/5/7, 3/4/1, 3/6/7, 4/5/2, 4/7/6, 5/7/8, 6/7/4}
			\path (\f) edge[->,draw] node[below, circle,draw=none,inner sep=.3mm] {\w} (\t);
	\end{scope}
}