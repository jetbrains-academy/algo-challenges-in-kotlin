\documentclass{article}
\usepackage{textcomp}
\usepackage[dvipsnames]{xcolor}
\usepackage{amsopn, amsmath, amssymb}
\usepackage{booktabs}

\newcommand{\myproc}[1]{\text{\textsc{\text{#1}}}}
\newcommand{\myvar}[1]{\text{\sl #1}}

\usepackage{tikz}
\usepackage{tikz-qtree}
\usepackage{pgfplots}

\usetikzlibrary{
	positioning, 
	matrix, 
	calc,
	arrows,
	shapes,
	fit,
	decorations,
	decorations.pathmorphing
}

\tikzset{
	every picture/.style={
		line width=.4mm,
		text=mc,
		draw=mc,
	},
}

\tikzstyle{arraynodes} = [
	minimum size=5mm, 
	inner sep=0mm,
	anchor=center,
	draw=mc,
	text=mc
]

\tikzstyle{array} = [
	draw=none,
	matrix of nodes,
	nodes={arraynodes},
	ampersand replacement=\&,
	line width=.3mm,
	row sep=-\pgflinewidth,
	column sep=-\pgflinewidth,
]

\tikzstyle{h}=[arraynodes, fill=hcfill]

\newcommand{\logo}{
	\path[use as bounding box] (-2.6, -2.6) rectangle (2.6, 2.6);
	\draw[rounded corners] (-2.5, -2.5) rectangle (2.5, 2.5);
}

\newenvironment{pseudocode}
	{\hrule\medskip}
	{\medskip\hrule}

\usepackage{fullpage}

\colorlet{mc}{white!70!black} % main color
\colorlet{hc}{yellow} % highlight color
\colorlet{mcfill}{mc!20!white}
\colorlet{hcfill}{hc!20!white}


\colorlet{one}{green!40}
\colorlet{two}{blue!30}
\colorlet{three}{red!30}
\colorlet{four}{yellow}

\begin{document}
Let $a_0 \le a_1 \le \dotsb \le a_7$ and $b_0 \le b_1 \le \dotsb \le b_7$
be~two sorted lists of~length $n=8$. We~would like to~find the median of~their sorted union $c_0 \le c_1 \le \dotsb \le c_{15}$,
i.e., the $8$-th element $c_7$. Note that $c_7$ follows seven elements and is~followed by~eight elements in~$c$.

Consider the following two prefixes of~$a$ and~$b$ of~total length $n+1=9$: $a_0 \le a_1 \le a_2$
and $b_0 \le b_1 \le b_2 \le b_3 \le b_4 \le b_5$. Assume that $a_2 \le b_5$. Can you prove that
some of $a_i$'s and $b_i$'s can be~discarded in~this case?

Intuitively, comparing the largest elements of~two prefixes of~total length $n+1$
allows one to~find the largest element among the union of~these prefixes
(that can be~put into the right half of~their sorted union).

Since $a_2 \le b_5$, we~conclude that
\[a_0 \le a_1 \le a_2 \le b_5 \le b_6 \le b_7 \, .\]
There are only six elements that can be~smaller than~$a_1$: $a_0, b_0, b_1, b_2, b_3, b_4$.
Hence, $a_1$ can be~among the first seven elements of~$c$. Thus, we~can discard 
$a_0$~and~$a_1$. Similarly, there are only six elements that can be~larger than $b_6$:
$b_7, a_3, a_4, a_5, a_6, a_7$. Hence, $b_6$ can be~among the last seven elements of~$c$.
Thus, we~can discard $b_6$~and~$b_7$.

This way, we~reduce the problem of~finding the median of~two sorted lists of~length eight
to~the problem of~finding the median of~two sorted lists of~length six.

Generalizing this toy example, consider two sorted lists $a_0 \le \dotsb \le a_{n-1}$ and $b_0 \le \dotsb \le b_{n-1}$ of~length~$n$. Our goal is~to~find the median of~their sorted union
$c_0 \le \dotsb \le c_{2n-1}$, that is, the element $c_{n-1}$. In~$c$, there are $n-1$ elements
before $c_{n-1}$ and $n$~elements after~$c_{n-1}$.
For a~parameter~$m$ to~be chosen later, consider the following two prefixes of~total length~$n+1$: $a_0 \le \dotsb \le a_m$ and $b_0 \le  \dotsb \le b_{n-m-1}$.
Compare the elements $a_m$ and $b_{n-m+1}$. 
If~$a_m \le b_{n-m-1}$,
	one can discard the first $m$~elements of~$a$ ($a_0, \dotsc, a_{m-1}$) 
	and the last $m$~element of~$b$ ($b_{n-m}, \dotsc, b_{n-1}$). Indeed, 
	the only elements that can be~smaller than $a_{m-1}$
	are $a_0, \dotsc, a_{m-2}$ and $b_0, \dotsc, b_{n-m}$ ($n-2$ elements),
	wheres the only elements that can be~larger than $b_{n-m}$
	are $b_{n-m+1}, \dotsc, b_{n-1}$ and $a_{m+1}, \dotsc, a_{n-1}$ ($n-2$ elements).
	This allows one to~reduce the problem to~finding the median of~two sorted lists of~length $n-m$.
The case $a_m > b_{n-m-1}$ is~treated similarly: one reduces the problem to~two sorted
lists of length $m+1$.

To~make sure that the problem size reduces significantly in~both cases, one takes $m=\lfloor \frac{n-1}{2} \rfloor$. This way, one halves the problem size at~every iteration. Hence, the number
of~recursive calls is~$O(\log n)$. The base case of~recursion is~$n=1$: one then returns the minimum of~the two elements. 
\end{document}
